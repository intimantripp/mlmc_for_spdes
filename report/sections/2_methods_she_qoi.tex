\subsection{Quantities of Interest for the Stochastic Heat Equation}\label{sec:QoI_for_SHE}

To validate our numerical schemes and analyse the performance 
of different MLMC coupling strategies, we compute expectations
of two distinct quantities of interest (QoI) derived from the solution 
of the SHE problem \eqref{eq:she_full_problem}.
The first is the squared amplitude of the SHE's Fourier Modes,
the second is a measure of the total energy of the 
system. We outline these here, derive the true values that
our MLMC version wishes to converge to, and then also 
determine what the weak error and variance decay values, $\alpha$ and 
$\beta$ respectively, we expect to observe. In the next section
we validate that our implementation converges to these results. 

\subsubsection{Squared Amplitude of Fourier Modes}

For the SHE problem \eqref{eq:she_full_problem}, 
the solution can be expanded in a Fourier sine series. This 
decomposition is justified 
as the sine functions are the eigenfunctions of the Laplacian 
operator with the given boundary conditions, and they also form a basis 
for decomposing the stochastic noise term via the Karhunen-Loéve theorem
(\cite{da2014stochastic}, Section 4.1).

\begin{equation*}
    u(t,x) = \sum_{n=1}^\infty u_n(t)\phi_n(x) \qquad \phi_n(x) = \sqrt{2}\sin (n\pi x)
\end{equation*}

where $u_n(t)$ is the $n$-th Fourier mode at time $t$ and $\phi_n$ our 
orthonormal sine basis functions. $u_n(t)$ is given by the $L^2$ 
inner product of $u$ with $\phi_n$:

\begin{equation*}
    u_n(t) = \int_0^1 u(t,x)\phi_k(x)\mathrm{d}x
\end{equation*}

thus our QoI, the expected squared Fourier Amplitude is given by:

\begin{equation*}
    \mathbb{E}\left[u_n(t)^2\right] = 
    \mathbb{E}\left[\left(\int_0^1 u(t,x)\phi_k(x)\mathrm{d}x\right)^2\right]
\end{equation*}

Our sample estimates of Fourier modes are obtained via the Discrete Inner Product,
as defined in \cite{suli2025nspdes}.

\begin{definition}[Discrete Inner Product]
    \label{def:discrete_inner_product}
    For two functions $V$ and $W$ defined at interior mesh-points 
    $x_i$ for $i = 1, \dots, N$ with spatial step size $h$
    \begin{equation*}
        (V,W)_h := \sum_{i = 1}^{N} h V_i W_i
    \end{equation*}
\end{definition}

An $\ell$-th level estimate of the squared amplitude of the $n$-th Fourier mode
at timestep $j$, $(\hat{u}_n^{(\ell,j)})^2$, we compute as:

\begin{equation*}
    (\hat{u}_n^{(\ell, j)})^2 = \left(U^j, \phi_n\right)_{\Delta x_\ell}^2
\end{equation*}

We derive first an analytic expression for $u_n$, 
and then the first and second moments of $u_n$ for validating our 
MLMC implementation is correct. We will also derive the 
analytic error we expect to observe and 
the expected variance decay. 
\newline

\textbf{Analytic Moments of Fourier Modes}

To obtain analytic solutions to the Fourier modes, we first express each term 
in the SHE governing equation as a Fourier sine series (an approach justified 
in \cite{strauss2007partial}, Chapter 5, and \cite{da2014stochastic}, Section 4.1).

$$
\begin{aligned} 
    u_t(t,x) &= \sum_{n=1}^\infty \frac{\mathrm{d}u_n(t)}{\mathrm{d}t}\phi_n(x) \\ 
    u_{xx}(t,x) &= \sum_{n=1}^\infty -(n\pi)^2 u_n(t)\phi_n(x) \\ 
    \xi(x,t) &= \sum_{n=1}^\infty \dot{B}_n(t) \phi_n(x) 
\end{aligned}
$$

where $\dot{B}_n(t)$ represents one-dimensional white noise for each mode.
Substituting these expansions into the SHE yields:
$$\sum_{n=1}^\infty \left( \frac{\mathrm{d}u_n(t)}{\mathrm{d}t} - (-(n\pi)^2 u_n(t)) 
- \dot{B}_n(t) \right) \phi_n(x) = 0$$ $$\sum_{n=1}^\infty \left( \frac{\mathrm{d}u_n(t)}{dt}
 + (n\pi)^2 u_n(t) - \dot{B}_n(t) \right) \phi_n(x) = 0
 $$

For
this infinite sum to be zero for all $x$, the coefficient of each basis
function must be zero independently. This decouples the partial differential 
equation into an infinite system of SDEs, one for each mode 
$u_n(t)$: 
$$\frac{\mathrm{d}u_n(t)}{\mathrm{d}t} + (n\pi)^2 u_n(t) = \dot{B}_n(t)$$

Writing this in differential form, where $\mathrm{d}B_n(t) = 
\dot{B}_n(t)\mathrm{d}t$ is the 
increment of a standard one-dimensional Brownian motion, 
gives:
$$
\mathrm{d}u_n(t) = -(k\pi)^2 u_n(t)\mathrm{d}t + \mathrm{d}B_n(t)
$$ 

This is the equation for an \textit{Ornstein-Uhlenbeck process}.
With a zero initial condition, $u_n(0)=0$, the solution to 
this SDE is a Gaussian process (\cite{oksendal2013stochastic}, Chapter 5.1). 
Its first two moments are well-known:

\begin{align}\label{eq:moments_of_fourier_modes}
\mathbb{E}[u_n(t)] &= 0 \\
\mathbb{V}[u_n(t)] &= \mathbb{E}[u_n(t)^2] = 
 \frac{1 - e^{-2(n\pi)^2 t}}{2(n\pi)^2} \label{eq:var_fourier_modes}
\end{align}

Thus,

\begin{align}
    u_n(t) &\sim \mathcal{N}(0, 
    \frac{1-e^{-2\lambda_n^2t}}{2\lambda_n^2}), 
    \quad \lambda_n = \pi^2 n^2
    \nonumber
    \\
    \mathbb{E}\left[u_n(t)^2\right] &= \frac{1 - e^{-2\lambda_n^2t}}{2\lambda_n^2}
    \label{eq:squared_amplitude_analytic}
\end{align}
\newline

\textbf{Weak Error and Variance Decay of Squared Amplitude of Fourier Modes}

We now establish the decay
rates for our MLMC estimators. We must analyse two key properties:
the weak error of the finite difference scheme and the rate 
of decay of the MLMC variance, $V_\ell$. We present 
two propositions and their accompanying proofs to formally 
derive these rates for our chosen QoI, the squared amplitude (variance)
of a single Fourier mode. 

\begin{proposition}[Weak Error for the Squared Amplitude of a Fourier Mode]
    \label{prop:weak_error_for_fourier_mode}
    The explicit finite difference scheme for the Stochastic Heat Equation
    governing the $n$-th Fourier mode, with time step $\Delta t$, has a weak error 
    in its stationary variance of 
    $\mathcal{O}(\Delta t)$. Consequently, for a finite difference 
    implementation of the SHE with a fixed CFL number such that 
    $\Delta t \propto (\Delta x)^2$, the error in the squared amplitude of the mode 
    is of order $\mathcal{O}((\Delta x)^2)$. 
\end{proposition}

Before presenting our proof, we conclude from it that
that, recalling Theorem \ref{theorem:mlmc_complexity}, 
we have a decay in the weak error of $\alpha = 2$. This follows from Proposition 
\ref{prop:weak_error_for_fourier_mode} as:

\begin{align*}
    \mathbb{E}[P_\ell - P] =& \mathcal{O}\left((\Delta x_\ell)^2\right)\\
                           =& \mathcal{O}\left((2^{-(\ell + 1)})^2\right) 
                           \quad (\Delta x_\ell = 2^{-(\ell+1)} \text{ for MLMC Implementation)}\\
                           =& \mathcal{O}\left( 2^{-2\ell}\right)
\end{align*}


\begin{proof}
The proof compares the exact variance of the
continuous Ornstein-Uhlenbeck 
(OU) process
with the variance of its finite difference approximation. 
Following from \eqref{eq:var_fourier_modes}, we can obtain that the 
stationary variance of the $n$-th Fourier mode is the following:

\begin{equation}\label{eq:stationary_variance}
    \lim_{t \to \infty} \mathbb{V}[u_n]
    := \mathbb{V}[u_n^{\infty}] = \frac{1}{2\lambda_n^2}
\end{equation}

Recalling our finite difference scheme in \eqref{eq:she_scheme},
we express it instead in matrix vector form, with $N$ 
internal points.

\begin{align}
    \mathbf{U}^j := 
    \begin{bmatrix}
        U_1^j \\
        \vdots \\
        U_N^j
    \end{bmatrix} 
    \in \mathbb{R}^N,
    \quad
    \mathbf{Z}^j := 
    \begin{bmatrix}
        Z_1^j \\
        \vdots \\
        Z_N^j
    \end{bmatrix}
    &\overset{\mathrm{i.i.d.}}{\sim} \mathcal{N}(0, \mathbf{I}_N),
    \quad
    \boldsymbol{\phi}_n := \sqrt{2}
    \begin{bmatrix}
        \sin\left( \frac{n \pi}{N+1} \cdot 1 \right) \\
        \vdots \\
        \sin\left( \frac{n \pi}{N+1} \cdot N \right)
    \end{bmatrix}
    \in \mathbb{R}^N, \\
    \mathbf{U}^{j+1} &= \mathbf{U}^j + \Delta t A \mathbf{U}^j
    + \sqrt{\frac{\Delta t}{\Delta x}} \mathbf{Z}^j \label{eq:scheme_matrix_vector}\\
    \text{where } A &= \frac{1}{(\Delta x)^2}
    \begin{bmatrix}
        -2 & 1 & & \\
        1 & -2 & 1  \\
        & \ddots & \ddots & \ddots \\
        & & 1 & -2
    \end{bmatrix}
\end{align}

We can equivalently express our finite difference estimates $\mathbf{U}^j$
as a finite series of Fourier modes along with our basis functions 
$\boldsymbol{\phi}$:

\begin{equation*}
    \boldsymbol{U}^j = \sum_{n=1}^N \hat{u}_n^j \boldsymbol{\phi}_n
\end{equation*}

where $\hat{u}_n^j$ represents our estimate of the $n$-th Fourier 
mode at time step $j$. We define the discrete inner product now 
based on the definition presented in \cite{suli2025nspdes}.

Our discrete basis functions are orthogonal under Definition \ref{def:discrete_inner_product}
Discrete Inner product (\cite{strang2007computational}, Chapter 2), i.e.

\begin{equation*}
    (\boldsymbol{\phi}_n, \boldsymbol{\phi}_m)_{\Delta x} 
    = \sum_{n=1}^N \Delta x \phi_{n,i} \phi_{m,i} = \delta_{n,m}
\end{equation*}

where $\delta_{n,m}$ is the Kronecker-delta function.
Applying the discrete inner product to our 
scheme \eqref{eq:scheme_matrix_vector} and examining each term:

\begin{align}
    \begin{split}
        (\mathbf{U}^{j+1}, \boldsymbol{\phi}_n)_{\Delta x} &= 
        (\mathbf{U}^j, \boldsymbol{\phi}_n)_{\Delta x} + 
        \Delta t (A \mathbf{U}^j, \boldsymbol{\phi}_n)_{\Delta x}
        \label{eq:inner_product_var_res}
        \\
        &\qquad + \sqrt{\frac{\Delta t}{\Delta x}} 
        (\mathbf{Z}^j, \boldsymbol{\phi}_n)_{\Delta x},
    \end{split}\\
    (\mathbf{U}^{j+1}, \boldsymbol{\phi}_n)_{\Delta x} &=
    \left(\sum_{k=1}^{N} \hat{u}_k^{j+1} \boldsymbol{\phi}_k, 
    \boldsymbol{\phi}_n\right)_{\Delta x} = 
    \hat{u}_n^{j+1}, 
    \nonumber
    \\
    (\mathbf{U}^j, \boldsymbol{\phi}_n)_{\Delta x} &= \hat{u}_n^j, 
    \nonumber
    \\
    \Delta t (A \mathbf{U}^j, \boldsymbol{\phi}_n)_{\Delta x} &= 
    \Delta t \Delta x (A \mathbf{U}^j)^T \boldsymbol{\phi}_n = 
    \Delta t \Delta x (\mathbf{U}^j)^T A \boldsymbol{\phi}_n
    \label{eq:eigenval_1}\\
    &= \Delta t \Delta x \lambda_n^{FD}(\mathbf{U}^j)^T \boldsymbol{\phi}_n 
    \nonumber
    \\
    &= \Delta t \lambda_n^{FD} (\mathbf{U}^j, \boldsymbol{\phi}_n)_{\Delta x}
    \nonumber
    \\
    &= \Delta t \lambda_n^{FD} \hat{u}_n^j
     \label{eq:eigenval_2}\\
     \text{where }\lambda_n^{FD}&= \frac{4}{(\Delta x)^2} 
     \sin^2(\frac{\pi n}{2(N+1)}),
     \nonumber
     \\
    \sqrt{\frac{\Delta t}{\Delta x}}
    (\mathbf{Z}^j, \boldsymbol{\phi}_n)_{\Delta x} &= 
    \sqrt{\frac{\Delta t}{\Delta x}} \sum_{k=1}^N
    \Delta x \mathbf{Z}_k^j(\boldsymbol{\phi}_n)_k 
    \nonumber
    \\
    &= 
    \sqrt{\Delta t \Delta x} \sum_{k=1}^N \mathbf{Z}_k^j
    (\boldsymbol{\phi_n})_k 
    \nonumber
    \\
    &:= \nu_n^j 
    \sim \mathcal{N}(0, \sigma_n^2),
    \label{eq:sum_of_norm_variables}
\end{align}

where in going from \eqref{eq:eigenval_1} to \eqref{eq:eigenval_2}
we have substituted the eigenvalues, $\lambda_n^{FD}$, of the matrix 
$A$ (\cite{strang2007computational}, Chapter 2) and applied the orthogonality
of the basis functions, and in 
\eqref{eq:sum_of_norm_variables} we have 
used that the sum of normal random variables is normally 
distributed.

We obtain that \eqref{eq:inner_product_var_res} is equal to:
\begin{equation}
    \hat{u}_n^{j+1} = (1 - \Delta t \lambda_n^{FD}) \hat{u}_n^j + 
    \nu_n^j
\end{equation}

We now determine what the variance of the random variable 
$\nu_n^j$ is:
\begin{align*}
    \mathbb{V}[\nu_n^j] &= \mathbb{V}\left[\sqrt{\Delta t \Delta x}
    \sum_{k=1}^N\mathbf{Z}_k^j(\boldsymbol{\phi}_n)^k\right] \\
    &=\Delta t \Delta x \sum_{k=1}^N \mathbb{V}[\mathbf{Z}_k^j]
    ((\boldsymbol{\phi}_n)_k)^2 \\
    &= \Delta t \sum_{k=1}^N \Delta x  ((\boldsymbol{\phi}_n)_k)^2\\
    &= \Delta t
\end{align*}

where we have used the orthogonality of the basis functions, 
$(\boldsymbol{\phi}_n, \boldsymbol{\phi}_m)_{\Delta x} = 
\delta_{n,m}$.

Taking the variance of equation 
\eqref{eq:inner_product_var_res} therefore becomes:

\begin{equation}\label{eq:var_of_relation}
    \mathbb{V}[\hat{u}_n^{j+1}] = (1-\Delta t \lambda_n^{FD})^2
    \mathbb{V}[\hat{u}_n^j] + \Delta t
\end{equation}

For the stationary variance, we have that 
$\mathbb{V}[\hat{u}_n^{j+1}] = 
\mathbb{V}[\hat{u}_n^{j}] = 
\mathbb{V}[\hat{u}_j^\infty]$. Solving for the stationary 
variance gives:

\begin{align}
\mathbb{V}[\hat{u}_j^\infty] 
&= \frac{\Delta t}{1 - 
(1-\Delta t \lambda_n^{FD})^2} = 
\frac{1}{2\lambda_n^{FD} - 
(\lambda_n^{FD})^2 \Delta t}
\label{eq:fourier_variance_stationary}
\\
&= \frac{1}{2\lambda_n^{FD}(1 - \frac{\Delta t \lambda_n^{FD}}{2})}\\
&= \frac{1}{2\lambda_n^{FD}}(1 + \frac{\Delta t \lambda_n^{FD}}{2}
+ \dots)\label{eq:using_taylor_series} \qquad 
\text{(By Taylor Series)}
\end{align}

Recalling that $\lambda_n^{FD} = \frac{4}{(\Delta x)^2}
\sin^2\left(\frac{\pi n}{2(N+1)}\right) = \frac{4}{(\Delta x)^2}
\sin^2\left(\frac{n \pi \Delta x}{2}\right)$, as $\Delta x = \frac{1}
{N+1}$, and expanding via another Taylor series yields:

\begin{align*}
    \lambda_n^{FD} &= \frac{4}{(\Delta x)^2}\left(\frac{n^2 
    \pi^2 (\Delta x)^2}{4} + \mathcal{O}((\Delta x)^2)\right) = 
    n^2\pi^2 + \mathcal{O}((\Delta x)^2) \\
    &= \lambda_n^2 + \mathcal{O}((\Delta x)^2)
\end{align*}

Substituting this into \eqref{eq:using_taylor_series} yields:

\begin{align*}
    \mathbb{V}[\hat{u}_j^\infty] &= 
    \frac{1}{2\lambda_n^{FD}}\left(1 + 
    \frac{\Delta t \lambda_n^{FD}}{2} + \dots\right) \\
    &= \frac{1}{2\left(\lambda_n^2 + \mathcal{O}(\Delta x^2)\right)}\left(1 + \frac{\Delta t \lambda_n^{FD}}{2} + \dots\right) \\
    &= \frac{1}{2 \lambda_n^2 \left(1 + \mathcal{O}(\Delta x^2)\right)}\left(1 + \frac{\Delta t \lambda_n^{FD}}{2} + \dots\right) \\
    &= \frac{1}{2 \lambda_n^2}\left(1 + \mathcal{O}(\Delta x^2)\right)\left(1 + \frac{\Delta t \lambda_n^{FD}}{2} + \dots\right) \\
    &= \frac{1}{2 \lambda_n^2}\left(1 + \mathcal{O}(\Delta x^2)\right)
    \label{eq:final_result}
\end{align*}

Finally therefore we obtain our result that the weak error
of our finite difference scheme estimate is :

\begin{equation*}
    |\mathbb{V}\left[u_n^\infty\right] - \mathbb{V}\left[\hat{u}_n^\infty]\right]| = |\frac{1}{2\lambda_n^2} - \frac{1}{2 \lambda_n^2}\left(1 + \mathcal{O}(\Delta x^2)\right)|
    = \mathcal{O}(\Delta x^2)
\end{equation*}
\end{proof}


Having established the weak error, we now turn to the MLMC variance decay. 
The following proposition will be used to validate the decay 
rate achievable in an MLMC implementation.

\begin{proposition}[Magnitude of MLMC Variance Decay for the Squared Amplitude 
    of Fourier Mode]
    \label{prop:variance_decay_fourier}
    Let $P_\ell = \frac{1}{M}\sum_{m=1}^M(\hat{u}_n^{(\ell,m)})^2$ 
    be the MC estimator for the squared amplitude 
    of the the $n$-th Fourier mode on level $\ell$. The magnitude of 
    the MLMC 
    variance, $V_\ell = \mathbb{V}[P_\ell - P_{\ell - 1}]$, is 
    determined by the correlation, $\rho$, between the underlying 
    discrete finite difference samples, 
    $\hat{u}_n^{(\ell, n)}$ and $\hat{u}_n^{(\ell-1, n)}$.
    \begin{itemize}
        \item If the processes are perfectly correlated ($\rho^2 = 1)$,
         $V_\ell = \mathcal{O}((\Delta x_\ell)^4)$.
        \item If the correlation is imperfect $(\rho^2 < 1)$, 
        $V_\ell = \mathcal{O}((\Delta x_\ell)^2)$.
    \end{itemize}
\end{proposition}

Following Proposition \ref{prop:variance_decay_fourier}, we conclude 
for the MLMC implementation:

\begin{align*}
    V_\ell = 
    \begin{cases}
        \mathcal{O}\left(2^{-4\ell}\right), \text{  thus } \beta = 4 \quad \text{if } \rho = \pm 1\\
        \mathcal{O}\left(2^{-2\ell}\right), \text{  thus } \beta = 2 \quad \text{otherwise}
    \end{cases}
\end{align*}

\begin{proof}
    To prove Proposition \ref{prop:variance_decay_fourier} we will expand the terms in
    $V_\ell$ and then 
    solve for each of them.
    \begin{align*}
    V_\ell &= \mathbb{V}[P_\ell - P^{\ell - 1}]\\
    &= \mathbb{V}[P_\ell] + 
        \mathbb{V}[P_{\ell-1}] - 
        2 \mathrm{Cov}\left((P_\ell, P_{\ell - 1}\right)
    \end{align*}

    We begin by expanding $\mathbb{V}[P_\ell]$:

    \begin{align*}
        \mathbb{V}[P_\ell] &= 
        \mathbb{V}\left[\frac{1}{M}\sum_{m=1}^M
        \left(\hat{u}_k^{(\ell, m)}\right)^2\right] \\
        &= \frac{1}{M}\mathbb{V}\left[ (\hat{u}_k^\ell)^2 \right]\\
        &= \frac{1}{M}\left(\mathbb{E}[(\hat{u}_k^\ell)^4] 
        - \left(\mathbb{E}[(\hat{u}_k^\ell)^2]\right)^2\right)
    \end{align*}

    This gives us two expectations to determine. 
    Starting with the latter of these, we recall that 
    Fourier modes are normally distributed with 
    zero mean.
    We therefore have the following:

    \begin{align*}
        \mathbb{V}[(\hat{u}_n^\ell)^2] &= 
        \mathbb{E}[(\hat{u}_n^\ell)^2] - 
        \underbrace{\mathbb{E}[\hat{u}_n^\ell]^2}_{=0}\\
        \mathbb{V}[\hat{u}_n^\ell] &= \mathbb{E}[(\hat{u}_n^\ell)^2]
    \end{align*}

    Referring to the earlier derivation of the variance 
    of the Fourier mode \eqref{eq:fourier_variance_stationary}:
    
    \begin{equation}\label{eq:fourier_mode_var}
        \mathbb{V}[\hat{u}_n^\ell] = \frac{1}{2 \lambda_n^{FD} - 
        \Delta t_\ell (\lambda_n^{FD})^2}.
    \end{equation}

    We denote by $\Delta x_\ell$ and $\Delta t_\ell$ the spatial
    and temporal step sizes in our finite difference scheme at level 
    $\ell$.
    These are related, for our explicit finite difference scheme,
    via $\Delta t = \mu (\Delta x)^2$, where $\mu$ is the CFL
    number. Recalling that $\lambda_n^{FD} = \frac{4}{(\Delta x)^4} 
    \sin^2\left(\frac{n \pi \Delta x}{2}\right) = \lambda_n - 
    \frac{n^4 \pi^4}{12}(\Delta x)^2 + \mathcal{O}((\Delta x)^4)$
    where $\lambda_n = n^2 \pi^2$, we have:

    \begin{align*}
        (\lambda_n^{FD})^2 &= (\lambda_n - 
        \underbrace{\frac{\pi^4 n^4}{12}}_{\text{=a}}(\Delta x_\ell)^2
         + \mathcal{O}((\Delta x)^4))^2\\
         &= \lambda_n^2 - 2 a \lambda_n (\Delta x)^2 + 
         \mathcal{O}((\Delta x)^4)
    \end{align*}

    Returning to \eqref{eq:fourier_mode_var}:

    \begin{align*}
        \mathbb{V}[\hat{u}_n^\ell] &= \frac{1}{2 \lambda_n^{FD} - 
        \Delta t_\ell (\lambda_n^{FD})^2} = 
        \frac{1}{2\lambda_n\left(1 - \frac{2a + \mu \lambda_n^2}{2\lambda_n}
        (\Delta x_\ell)^2 + \mathcal{O}\left((\Delta x_\ell)^4\right)\right)} \\
        &= \frac{1}{2 \lambda_n} \left(1 + 
        \frac{2a + \mu \lambda_n^2}{2 \lambda_n} (\Delta x_\ell)^2
        + \mathcal{O}\left((\Delta x_\ell)^4\right) \right)
        \qquad \text{(Taylor Series)} \\
        &= \frac{1}{2\lambda_n} + 
        \underbrace{\frac{2a + \mu \lambda_n^2}{4\lambda_n^2}}_{=b}(\Delta x_\ell)^2
        + \mathcal{O}\left((\Delta x_\ell)^4\right)
        \\
        &= \frac{1}{2\lambda_n} + 
        b(\Delta x_\ell)^2
        + \mathcal{O}\left((\Delta x_\ell)^4\right) := \sigma_\ell^2
    \end{align*}

    We next express $\mathbb{V}[\hat{u}_n^{\ell-1}]$ in terms of 
    $\sigma_\ell^2$. This is straightforward via our geometric relationships
    between spatial steps at adjacent levels, 
    \eqref{eq:she_discrete_relations}.

    \begin{equation}\label{eq:sigma_ell_1_var}
        \sigma_{\ell-1}^2 = \frac{1}{2\lambda_n} + 4b(\Delta x)^2 + 
        \mathcal{O}\left((\Delta x)^4\right)
    \end{equation}

    We now determine what $\mathbb{E}\left[(\hat{u}_n^\ell)^4\right]$ is.
    Again, as $\hat{u}_n^\ell$ is normally distributed with zero mean, we use
    the standard identity that $\mathbb{E}[X^4] = 3 \sigma^4$ for 
    a normally distributed random variable $X$ with zero mean.
    This yields:

    \begin{align*}
        \mathbb{V}[(\hat{u}_n^\ell)^2] &= 
        \mathbb{E}[(\hat{u}_n^\ell)^4] + 
        \left(\mathbb{E}[(\hat{u}_n^\ell)^2]\right)^2 \\
        &= 3\sigma_l^4 - \sigma_l^4 = 2\sigma_l^4
    \end{align*}

    Finally, we obtain that:

    \begin{equation}\label{eq:P_var_estimates}
        \mathbb{V}[P_\ell] = \frac{2\sigma_l^4}{M}, \quad 
        \mathbb{V}[P_{\ell-1}] = \frac{2\sigma_{\ell-1}^4}{M}.
    \end{equation}

    We now examine $\mathrm{Cov}\left(P_\ell, P_{\ell-1}\right)$.

    \begin{equation*}
        \mathrm{Cov}(P_l, P_{l-1}) = 
        \frac{1}{M}\mathrm{Cov}(\hat{u}_l^2, \hat{u}_{l-1}^2) 
        = \frac{1}{M}\mathbb{E}[(\hat{u}_n^\ell \hat{u}_n^{l-1})^2] 
        - \mathbb{E}[(\hat{u}_n^\ell)^2]\mathbb{E}[(\hat{u}_n^{l-1})^2])
    \end{equation*}

    We already know $\mathbb{E}[(\hat{u}_n^\ell)^2] = \sigma_\ell^2$.
    To solve for the first term, we use Isserlis's Theorem 
    \cite{isserlis1918formula}, another 
    formula relating properties of normally distributed random variables.

    This gives us:
    \begin{align*}
        \mathbb{E}\left[(\hat{u}_n^\ell)^2(\hat{u}_n^{\ell-1})^2\right] 
        &= \mathbb{E}[(\hat{u}_n^{\ell})^2]\mathbb{E}[(\hat{u}_n^{\ell-1})^2]
         + 2\mathbb{E}[\hat{u}_n^\ell \hat{u}_n^{\ell-1}]^2 \\
        &= \sigma_\ell^2 \sigma_{\ell-1}^2 + 
        2\mathbb{E}[\hat{u}_n^\ell \hat{u}_n^{\ell-1}]^2
    \end{align*}

    We know that 
    $$
    \mathrm{Cov}(\hat{u}_n^\ell, \hat{u}_n^{\ell-1})
    = \mathbb{E}[\hat{u}_n^\ell \hat{u}_n^{\ell-1}] - 
    \underbrace{\mathbb{E}[\hat{u}_n^\ell]\mathbb{E}[\hat{u}_n^{\ell-1}]}_{=0}
    =\rho \sigma_l \sigma_{\ell-1}
    $$

    and so we now obtain that:

    \begin{equation}\label{eq:cov_fourier_var}
        \mathrm{Cov}(P_\ell, P_{\ell-1}) = 
        \frac{1}{M}\left(\sigma_\ell^2 \sigma_{\ell-1}^2 + 
        2\rho^2\sigma_\ell \sigma_{\ell-1} - 
        \sigma_\ell^2 \sigma_{\ell-1}^2\right)
        = \frac{2\rho^2\sigma_\ell^2\sigma_{\ell-1}^2}{M}
    \end{equation}

    Putting together \eqref{eq:P_var_estimates} and 
    \eqref{eq:cov_fourier_var} into $V_\ell$:

    \begin{equation}\label{eq:nearly_there_fourier_var}
        V_l = \mathbb{V}[P_l] + \mathbb{V}[P_{l-1}] - 
        2\mathrm{Cov}(P_l, P_{l-1}) = 
        \frac{2}{M}(\sigma_l^4 + \sigma_{l-1}^4 -
        2\rho^2 \sigma_l^2 \sigma_{l-1}^2)
    \end{equation}

    We expand each of these terms, also setting 
    $\frac{1}{2\lambda_n} = \sigma^2$, as it simplifies the expansion and 
    also corresponds to the stationary variance of the
    continuous Fourier mode \eqref{eq:var_fourier_modes}:

    \begin{align*}
        \sigma_l^4 &= \left(\sigma^2 + b (\Delta x_\ell)^2 + 
        \mathcal{O}\left((\Delta x_\ell)^4\right)\right)^2 = 
        \sigma^4 + 
        2 \sigma^2 b (\Delta x_\ell)^2 + \mathcal{O}\left((\Delta x_\ell)^4\right)
        \\
        \sigma_{l-1}^4 &= \left(\sigma^2 + 4b (\Delta x_\ell)^2 + 
        \mathcal{O}\left((\Delta x_\ell)^4\right)\right)^2 =
        \sigma^4 + 8\sigma^2 b(\Delta x_\ell)^2 + 
        \mathcal{O}\left((\Delta x_\ell)^4\right)
        \\
        \sigma_l^2 \sigma_{l-1}^2 &= \left(\sigma^2+ b(\Delta x_\ell)^2 + 
        \mathcal{O}\left((\Delta x_\ell)^4\right)\right)(\sigma^2
        + 4b(\Delta x_\ell)^2 + \mathcal{O}\left((\Delta x_\ell)^4\right))\\
        &= \sigma^4 + 5\sigma^2b(\Delta x_\ell)^2 + 
        \mathcal{O}\left((\Delta x_\ell)^4\right)
    \end{align*}

    Finally, substituting these terms into \eqref{eq:nearly_there_fourier_var}
    results in:
    \begin{align*}
        \sigma_\ell^4 - \sigma_{\ell-1}^4 - 2\rho^2 \sigma_\ell^2 
        \sigma_{\ell-1}^2 &= 2 \sigma^4(1 - \rho^2) + 10 \sigma^2b 
        (\Delta x_\ell)^2(1 - \rho^2) + 
        \mathcal{O}\left((\Delta x_\ell)^2\right) \\
        &= \begin{cases}
        \mathcal{O}\left((\Delta x_\ell)^4\right) \quad \text{if } 
        \rho = \pm 1 \\
        \mathcal{O}\left((\Delta x_\ell)^2\right) \quad \text{otherwise}
        \end{cases}
    \end{align*}

\end{proof}


\subsubsection{System Energy}

A fundamental property of the solution is its total energy, which is defined 
as the squared $L^2$-norm of the solution $u(t,x)$ at a final time $T$. The 
continuous form of this quantity if given by:

\begin{equation}\label{eq:system_energy}
    Q_{\text{energy}} = \int_0^1 (u(T,x))^2 \mathrm{d}x
\end{equation}

To derive the expected energy, $\mathbb{E}[Q_{\text{energy}}]$, 
we again express the solution 
$u(t,x)$ in its Fourier sine series
$u(t,x) = \sum_{n=1}^\infty u_n(t)\phi_n(x)$. Substituting this into 
\eqref{eq:system_energy}:

\begin{align*}
    \int_0^1 (u(T,x))^2 \mathrm{d}x &= 
    \int_0^1\left(\sum_{n=1}^\infty u_n(t)\phi_n(x)\right)
    \left(\sum_{n=1}^\infty u_n(t)\phi_n(x)\right) \mathrm{d}x \\
    &= \sum_{n=1}^\infty \sum_{m=1}^\infty u_n(t) u_m(t) (\phi_n, \phi_m) \\
    &= \sum_{n=1}^\infty (u_n(t))^2
\end{align*}

Taking the expectation:

\begin{align*}
    \mathbb{E}\left[\int_0^1 (u(T,x))^2 \mathrm{d}x\right] &= 
    \mathbb{E}\left[\sum_{n=1}^\infty (u_n(t))^2\right] \\
    &= \sum_{n=1}^\infty \mathbb{E}\left[(u_n(t))^2\right] \\
    &= \sum_{n=1}^\infty \frac{1-e^{-2\lambda_n t}}{2\lambda_n}
    \qquad \text{via } \eqref{eq:var_fourier_modes}\\
    &= \frac{1}{12} - \sum_{n=1}^\infty \frac{e^{-2n^2 \pi^2 t}}{2 n^2 \pi^2}\\
    &\approx \frac{1}{12} \quad \text{for sufficiently large } t
\end{align*}

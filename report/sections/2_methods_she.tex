\section{Stochastic Heat Equation Implementation}\label{sec:she_implementation}
\subsection{Problem Specification and Finite Difference Scheme}\label{sec:she_scheme_mlmc_imp}

We consider the one-dimensional SHE on a unit interval $[0,1]$ with
homogenous Dirichlet boundary conditions and a zero initial condition.
This represents the simplest setting for a parabolic SPDE. 
The problem is formally defined as finding the real-valued 
function $u(t,x)$ that satisfies:

\begin{subequations} \label{eq:she_full_problem}
\begin{align}
    \frac{\partial u(t,x)}{\partial t} - \frac{\partial^2 u(t,x)}{\partial x^2} &= \xi(x,t),
    \qquad &&\text{for } (t,x) \in (0, T] \times (0,1) \label{eq:she_pde} \\
    u(0,x) &= 0, \qquad &&\text{for } x \in [0,1] \label{eq:she_ic} \\
    u(t,0) &= 0, \quad u(t,1) = 0, \qquad &&\text{for } t \in (0, T]. \label{eq:she_bc}
\end{align}
\end{subequations}

To solve \eqref{eq:she_full_problem} numerically, we employ
an explicit finite difference scheme
obtained using a finite volume approach \cite{suli2025nspdes} where \eqref{eq:she_pde}
is integrated over small, discreet space-time control volumes
and the resulting integral terms then approximated.
This scheme is:

\begin{align}
    U_j^{n+1} &= U_j^n + \frac{\Delta t}{(\Delta x)^2} 
    (U_{j+1}^n - 2U_j^n + U_{j-1}^n) + \Delta W_j^n, \label{eq:she_scheme} \\
    \text{where} \quad \Delta W_j^n &= \sqrt{\frac{\Delta t}{\Delta x}} 
    Z_j^n \quad \text{and} \quad Z_j^n \overset{\mathrm{i.i.d.}}{\sim} \mathcal{N}(0,1). \nonumber
\end{align}

The derivation of this scheme now follows.

First, we define a uniform grid. The spatial domain 
$[0,1]$ is discretised into $J$ intervals of width 
$\Delta x = 1 / J$, with grid points $x_j = j \Delta x$ for 
$j = 0, 1, \dots J$. Similarly, let the time interval 
$[0, T]$ be discretised into $N$ steps of size $\Delta t 
= T / N$, with the time points $t_n = n \Delta t$ for 
$n = 0, 1, \dots N$. 
Our discrete approximations of $u$ are denoted $U_j^n \approx 
u(t_n, x_j)$ with $U_j^0 = 0$, $U_0^n = U_J^n = 0$ 
capturing our initial and boundary conditions respectively.

We integrate \eqref{eq:she_pde} over a control volume $C_j^n = 
[x_j - \frac{\Delta x}{2}, x_j + \frac{\Delta x}{2}] \times [t_n, t_n+1]$:

\begin{equation}\label{eq:she_integration}
    \int_{t_n}^{t_{n+1}} 
    \int_{x_j-\frac{\Delta x}{2}}^{x_j+\frac{\Delta x}{2}} 
    \frac{\partial u}{\partial t} \,\mathrm{d}x\mathrm{d}t -
    \int_{t_n}^{t_{n+1}} 
    \int_{x_j-\frac{\Delta x}{2}}^{x_j+\frac{\Delta x}{2}} 
    \frac{\partial^2 u}{\partial x^2} \,\mathrm{d}x\mathrm{d}t = 
    \int_{t_n}^{t_{n+1}} 
    \int_{x_j-\frac{\Delta x}{2}}^{x_j+\frac{\Delta x}{2}} 
    \xi(t,x) \,\mathrm{d}x\mathrm{d}t
\end{equation}


We then approximate each term in this equation. Focussing first on 
the time derivative term on the LHS:

\begin{subequations}
    \begin{align*}
        \int_{t_n}^{t_{n+1}} 
        \int_{x_j-\frac{\Delta x}{2}}^{x_j+\frac{\Delta x}{2}} 
        \frac{\partial u}{\partial t} \,\mathrm{d}x\mathrm{d}t 
        &= \int_{x_j-\frac{\Delta x}{2}}^{x_j+\frac{\Delta x}{2}} 
        \left[ u(t_{n+1}, x) - u(t_n, x) \right] \,\mathrm{d}x 
        && \parbox[t]{3.5cm}{\raggedright\small 
        (Fundamental Theorem of Calculus (FTOC))} \\
        &\approx \Delta x (U_j^{n+1} - U_j^n)
        && \text{(Midpoint Rule)}
    \end{align*}
\end{subequations}

Similarly, for the spatial derivative term:

\begin{subequations}
\begin{align*}
    \int_{t_n}^{t_{n+1}} 
    \int_{x_j-\frac{\Delta x}{2}}^{x_j+\frac{\Delta x}{2}} 
    \frac{\partial^2 u}{\partial x^2} \,\mathrm{d}x\mathrm{d}t  
    &=\\ 
    &\int_{t_n}^{t_{n+1}} 
    \left[ \frac{\partial u}{\partial x}\left(t, x_j + 
    \frac{\Delta x}{2}\right) - 
    \frac{\partial u}{\partial x}\left(t, x_j - 
    \frac{\Delta x}{2}\right) \right] \,\mathrm{d}t 
    && \parbox[t]{3.5cm}{\raggedright\small 
    (FTOC)} \\
    &\approx \frac{\Delta t}{\Delta x} \left[ 
        U_{j+1}^n - 2U_j^n + 
        U_{j-1}^n \right] 
    && \parbox[t]{3.5cm}{\raggedright\small 
    (Midpoint Rule and Central Differences)}
\end{align*}
\end{subequations}

Finally, the forcing term on the RHS, via equation \eqref{eq:white_noise_result}, is equal to

\begin{equation*}
\int_{t_n}^{t_{n+1}} \int_{x_j-\frac{\Delta x}{2}}^{x_j+\frac{\Delta x}{2}} \xi(t,x) \,\mathrm{d}x\mathrm{d}t = \sqrt{\Delta x \Delta t} Z_j^n
\end{equation*}

where $Z_j^n$ are indepenent and identically distributed standard normal random variables.
Finally, we substitute the discrete approximations for each of the three terms back into the 
integral equation \eqref{eq:she_integration}. This yields the following relation:
\begin{equation*}
    \Delta x (U_j^{n+1} - U_j^n) = 
    \frac{\Delta t}{(\Delta x)^2} \Delta x (U_{j+1}^n - 2U_j^n + U_{j-1}^n) + 
    \sqrt{\Delta t \Delta x} Z_j^n
\end{equation*}

Rearranging yields the scheme shown in equation \eqref{eq:she_scheme}.
This scheme is used to propagate the solution forward in time. It is known to 
be conditionally stable, requiring the Courant-Friedrichs-Lewy (CFL) condition, 
$\frac{\Delta t}{(\Delta x)^2} \le \frac{1}{2}$, to be satisfied for convergence, 
similar to the corresponding deterministic heat equation \cite{suli2025nspdes}. 
A proof of this is given in the Appendix (CITE APPENDIX HERE).

\subsection{Stochastic Heat Equation - MLMC Implementation}

We now describe how scheme \eqref{eq:she_scheme} is implemented for our MLMC
algorithm \ref{alg:mlmc_detailed}. This applies for a generic quantity of interest
(QoI), $P$. The QoIs tested for this investigation are outlined in section 
\ref{sec:QoI_for_SHE}.

For each level $\ell$, we divide the spatial domain into $n_{\ell} = 2^{l+1}$ subdivisions
resulting in a spatial step size $\Delta x_\ell = 1 / n_\ell$. We fix CFL number $\lambda = 
\frac{\Delta t}{(\Delta x)^2} = 0.25$. This yields a timestep of 
$\Delta t_\ell = \lambda (\Delta x_\ell)^2$. This means that for any two 
consecutive levels the refinement ratios are related via

\begin{equation}\label{eq:she_discrete_relations}
    \Delta x_{\ell - 1} = 2\Delta x_\ell, \qquad \Delta t_{\ell - 1} = 4 \Delta t_\ell.
\end{equation}

For each level $\ell$, to obtain a sample at that level we evolve scheme \eqref{eq:she_scheme} 
with discretisations $\Delta x_\ell$ and $\Delta t_\ell$ the desired number of time steps, 
and then compute our discrete estimate of $P$, $P_\ell$. 

We highlight that, given these refinement ratios, an additional level 
of refinement results in 8 times as many cells in the grid. Therefore, 
we anticipate the cost to scale as $C_\ell \propto 2^{3\ell}$, implying a 
cost growth rate of $\gamma = 3$.

For noise coupling, we investigate 3 strategies: Right-Most Nearest Neighbours (NN)
used in \cite{cornalba2025multilevel}, Central Coupling (CC) and a Finite Element (FE)
based coupling. We outline each of these now. We follow the convention used in 
\cite{giles2015multilevel} for coupling, describing the $\ell$-th level as 
the \textit{fine} level and the $\ell - 1$-th level as the \textit{coarse} level.
\newline

\textbf{Right-Most Nearest Neighbours Coupling}

For the fine level's noise generation, at each interior fine grid point, $j$, and each fine 
time step, $n$, an independent noise increment is generated:

\begin{equation*}
    \Delta W_{j,f}^n = \sqrt{\frac{\Delta t_f}{\Delta x_f}} Z_j^n, \qquad \text{where } 
    Z_j^n \overset{\mathrm{i.i.d}}{\sim} \mathcal{N}(0,1)
\end{equation*}

The coarse level's noise at a grid point, $k$,
corresponding to fine grid point $2j$,
over a coarse time step, $m$, is constructed by 
aggregating the underlying fine noise. We sum 
the noise from the corresponding fine grid point
and its immediate right-hand Neighbour
$2j + 1$. This sum is accumulated over 
the 4 underlying fine timesteps 
and then rescaled.

\begin{equation*}
    \Delta W_{k,c}^m = \frac{1}{2} \sum_{n = 4m}^{4m+3}
    \left(\Delta W_{2j,f}^n + W_{2j+1, f}^n \right)
\end{equation*}

The rescaling factor $\frac{1}{2}$ is essential to 
ensure that the coarse noise $\Delta W_{k,c}^m$
has the correct statistical variance of 
$\frac{\Delta t_c}{\Delta x_c}$ required for 
coarse grid simulation.

We highlight that this method discards the final 
interior noise increment from the fine grid
point. We anticipated this having some 
detrimental effect compared to 
other coupling strategies, as 
this clearly leads to imperfect correlation
between fine and coarse samples.
This motivated the following two strategies.
\newline

\textbf{Central Coupling}

As an alternative to the asymmetric NN method, we
propose a centred coupling strategy. This aims 
to create a more symmetric correlation between the fine 
and coarse grids by defining the fundamental source 
of randomness on a "half-cell" refinement of the 
spatial grid. From this common source of randomness, 
the noise increments for both the fine and coarse 
grids are constructed. 

We divide each internal fine grid cell 
$[x_{j}-\frac{\Delta x_f}{2}, 
x_{j}+\frac{\Delta x_f}{2}]
\times [t_n, t_n + \Delta t_f]$ into
two half-cells of width $\frac{\Delta x_f}{2}$.
For each of these half-cells and for each fine time
step $n$, we generate an independent, fundamental 
noise increment $\zeta$. Let $\zeta_{j,L}^n$ and 
$\zeta_{j,R}^n$ be the half-cell noises on the left 
and right halves of the $j$-th fine grid point
during the $n$-th time step. These are 
i.i.d Gaussian random variables with variance 
equal to the area of the half-cell:

\begin{equation*}
    \zeta_{j,L}^n, \zeta_{j, R}^n 
    \overset{\mathrm{i.i.d}}{\sim} 
    \mathcal{N}(0, \frac{\Delta x_f \Delta t_f}{2})
\end{equation*}

The noises for the fine finite difference 
scheme are then constructed by aggregating 
these fundamental half-cell noises.

\begin{equation*}
    \Delta W_{j,f}^n = \frac{1}{\Delta x_f} 
    (\zeta_{j,L}^n + \zeta_{j,R}^n)
\end{equation*}

where the $\frac{1}{\Delta x_f}$ coefficient 
ensures $\Delta W_{j,f}^n$ has the correct variance. 

Similarly to the NN coupling, the coarse noise 
is constructed as an aggregation of 
the underlying fine noises accummulated over the
four fine timesteps that constitute a single
coarse time step, equivalent to using the 
exact underlying 16 half-cell noises.

\begin{equation*}
    \Delta W_{k, c}^n = \frac{1}{\Delta x_c}
    \sum_{n=4m}^{4m+3} \left(\Delta W_{2j,f}^n + 
    \Delta W_{2j+1}^n\right) = \frac{1}{\Delta x_c}
    \sum_{n=4m}^{4m+3} \left(\zeta_{2j,L}^n + 
    \zeta_{2j, R}^n + \zeta_{2j+1,L}^n + 
    \zeta_{2j+1,R}^n\right)
\end{equation*}

Again, the scaling factor
$\frac{1}{\Delta x_c}$ ensures the resulting
noises have the correct variances required on 
the coarse grid. 
By further discretising the spatial grid, 
this method aimed to better align the noises 
used for each grid point in both the coarse
and fine grids, unlike the NN method, ensuring
no underlying fine noise was discarded in 
constructing coarse noise.
\newline

\textbf{Finite Element Coupling Method}

The third coupling strategy is derived from a Galerkin Finite Element Method
(FEM) \cite{suli2025fe} spatial discretisation of the SHE. We transform 
the infinite-dimensional SPDE into a finite-dimensional system of SDEs, 
which in turn defines a structure of the discrete noise and coupling between 
levels.

We begin by formulating a weak form of the SHE. We multiply \eqref{eq:she_pde} by 
a sufficiently smooth test function $\phi(x)$ and integrate over the spatial domain 
$D = [0,1]$. Applying integration by parts yields:

\begin{equation}\label{eq:she_differential_form}
    (\mathrm{d}u,\phi) + (u_x, \phi_x)\mathrm{d}t = (\mathrm{d}W(t), \phi)
\end{equation}

where $(\cdot, \cdot)$ denotes the $L^2$ inner product and  
$\mathrm{d}W(t) = \xi(t,x)\mathrm{d}t$ represents Brownian
motion.

The Galerkin method seeks an approximate solution, $U(t,x)$ within a finite-dimensional
subspace spanned by a set of basis functions \cite{suli2025fe}. For this problem, 
we use the standard piecewise linear ``hat'' basis functions $\phi_j(x)$, defined on a 
uniform grid with spacing $h$:

\begin{equation*}
    \phi_j(x) = \max(0, 1 - |x-x_j|/ h).
\end{equation*}

The approximate solution is written as

\begin{equation*}
    U(t,x) = \sum_{j=1}^{J-1}U_j(t)\phi_j(x).
\end{equation*}

Here, the $U_j(t)$ are the time-dependent coefficients represents the solution's value at
the spatial nodes $x_j$. By requiring the weak form to hold for every basis function 
in this space, we obtain a system of SDEs for the vector of coefficients $\mathbf{U}(t)$. 

The system of SDEs can be written in matrix form as:

\begin{equation}\label{eq:fe_sdes_system}
    M \mathrm{d}\mathbf{U}(t) + K \mathbf{U}(t)\mathrm{d}t = \mathrm{d}W(t)
\end{equation}

where $M$ is a tri-diagonal matrix with elements $M_{ij} = (\phi_i, \phi_j)$. 
For the hat basis, this gives

\begin{equation*}
    M_{i,i} = \frac{2}{3}h, \qquad M_{i, i\pm 1} = \frac{1}{6}h.
\end{equation*}

$K$ is a tridiagonal matrix with entries $K_{ij} =(\phi'_i, \phi'_j)$. This gives

\begin{equation*}
    K_{i,i} = \frac{2}{h}, \qquad K_{i, i\pm 1} = - \frac{1}{h}
\end{equation*}

$\mathrm{d}W(t)$ is a vector of normal increments. To determine their variance and covariance, 
we can derive the following covariance function:

\begin{equation*}
    \mathbb{E}[(f, \mathrm{d}W)(g, \mathrm{d}W)] = (f,g) \mathrm{d}t
\end{equation*}

for arbitrary spatial functions $f$ and $g$. This follows from Definition \ref{def:whitenoise}
by considering test functions $\phi_f$ and $\phi_g$:

\begin{align*}
    \phi_f(s,x) = f(x)\mathbf{1}_{[t, t+\mathrm{d}t]}(s)\\
    \phi_g(s,x) = g(x)\mathbf{1}_{[t, t+\mathrm{d}t]}(s)
\end{align*}

where $\mathbf{1}_{[t, t+\mathrm{d}t]}(s)$ is an indicator function which is 1 for
$s \in [t, t + \mathrm{dt}]$, $0$ otherwise. Th abstract random variable 
$W(\phi_f)$ from Definition \ref{def:whitenoise} now represents the noise tested 
against the spatial function $f(x)$


\begin{align*} 
    \mathbb{E}[(f, \mathrm{d}W)(g, \mathrm{d}W)] &= \mathbb{E}[W(\phi_f)W(\phi_g)] \\ 
    &= (\phi_f, \phi_g) \\ &= \int_0^T \int_D \phi_f(s, \mathbf{x}) \phi_g(s, \mathbf{x})
     \,d\mathbf{x}ds \\ &= \int_0^T \int_D \left( f(\mathbf{x}) \mathbf{1}_{[t, t+\mathrm{d}t]}(s) \right) 
     \left( g(\mathbf{x}) \mathbf{1}_{[t, t+\mathrm{d}t]}(s) \right) \,d\mathbf{x}ds \\
    &= \left( \int_D f(\mathbf{x})g(\mathbf{x})\,d\mathbf{x} \right)
     \times \left( \int_0^T (\mathbf{1}_{[t, t+\mathrm{d}t]}(s))^2 \,ds \right)  \\
    &= \left(f,g\right) \mathrm{d}t
\end{align*} 

The vector of normal increments have the following expectations:

\begin{align*}
    \mathbb{E}[\mathrm{d}W_i \mathrm{d}W_j] = \mathbb{E}[(\mathrm{d}W, \phi_i) 
    (\mathrm{d}W, \phi_j)] = (\phi_i, \phi_j)\mathrm{d}t\\
    \mathbb{E}[\mathrm{d}W_i^2] = \frac{2}{3}h \mathrm{d}t, \qquad 
    \mathbb{E}[\mathrm{d}W_i \mathrm{d}W_{i\pm 1}] = \frac{1}{6} h \mathrm{d}t
\end{align*}

A common simplification known as mass lumping is used, where
matrix $M$ is replaced by the diagonal matrix $hI$. 
Applying this to \eqref{eq:fe_sdes_system} yields the scheme:

\begin{equation*}
    h\mathbf{U}^{n+1} = h\mathbf{U}^n - K \mathbf{U}^n \Delta t + \Delta W^n
\end{equation*}

where the discrete noise vector $\Delta W^n$ has the covariance structure 
$M \Delta t$.

The coupling between a fine level $\ell$ and a coarse level $\ell-1$ is derived 
from the relationship between the FEM basis functions. A coarse grid basis 
function, $\phi_k^c$, can be expressed as a linear combination of the fine grid basis functions:

\begin{equation*}
    \phi_k^c(x) = \frac{1}{2} \phi_{2k-1}^f(x) + \phi_{2k}^f(x) + \frac{1}{2}\phi_{2k+1}^f(x).
\end{equation*}

This provide a natural way to construct the coarse noise from the fine noise. The coarse
noise increment at a coarse node $k$, $\Delta W_{k,c}^m$, is constructed by applying the same 
linear weighting to the fine noise increments over the corresponding time steps: 

\begin{equation}\label{eq:fe_coupling_eqn}
    \Delta W_{k,c}^m = \frac{1}{2} \sum_{n=4m}^{4m+3} \left( 
        \frac{1}{2}\Delta W_{2k-1,f}^n + \Delta W_{2k,f}^n + 
        \frac{1}{2}\Delta W_{2k+1,f}^n \right).
\end{equation}

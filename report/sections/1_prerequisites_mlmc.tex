\section{Monte Carlo and Multilevel Monte Carlo}\label{sec:intro_mlmc}

The primary objective in many applications involving SPDEs is not to find 
a single, particular solution, but rather to compute the expected value of a quantity of 
interest that depends on the solution. For example, we may 
want to find the average temporature at a specific point in a domain 
governed by the stochastic heat equation, or the average particle density in a
system described by the Dean-Kawasaki equation.

Let $P$ represent such a quantity of interest, which we assume to be a real-valued 
random variable. Our goal throughout this section is to develop an efficient numerical
method for estimating its expectation, $\mathbb{E}[P]$. Since 
analytical expressions for $\mathbb{E}[P]$ are rarely available in this context, we must 
turn to computational methods. The most fundamental of these is defined below.

\begin{definition}[Monte Carlo Estimator]\label{def:mc_estimator}
    Let $\{P^{(n)}\}$ for $n = 1, \dots, N$ be a set of $N$ independent and identically
    distributed samples of a random variable $P$. The standard Monte Carlo estimator,
    $\hat{P}_{MC}$, of the expectation $\mathbb{E}[P]$ is the sample mean:
    \[
    \hat{P}_{MC} = \frac{1}{N} \sum_{n=1}^N P^{(n)}
    \]
\end{definition}

By linearity of the expectation operator, the MC estimator in Definition \ref{def:mc_estimator} 
is unbiased. The accuracy of an estimator is typically estimated via its Mean Squared Error (MSE) and 
Root Mean Square Error (RMSE), which we define now.

\begin{definition}[Mean Squared Error and Root Mean Squared Error]\label{def:mse_rmse}
    Let $P$ be a fixed unknown quantity and let $\hat{P}$ be an estimator for $P$. 
    The \textbf{Mean Square Error (MSE)} of the estimator is the expected value of 
    the squared error:
    \[
    \text{MSE}(\hat{P}) = \mathbb{E}[(\hat{P} - P)^2]
    \]
    This can be decomposed into terms representing the estimator's variance and squared bias:
    \[
    \text{MSE}(\hat{P}) = \underbrace{\mathbb{V}[\hat{P}]}_{\text{Variance of estimator}}
    + \underbrace{(\mathbb{E}[\hat{P}] - \mathbb{E}[P])^2}_{\text{Bias}^2}
    \]
    The \textbf{Root Mean Square Error (RMSE)} is the square root of the MSE.
\end{definition}

Since the standard MC estimator is unbiased, its bias term is zero. 
Its MSE is therefore equal to its variance:

\begin{equation*}
    \text{MSE}(\hat{P}_{MC}) = \mathbb{V}[\hat{P}_{MC}] = \frac{1}{N}\mathbb{V}[P]
\end{equation*}

The framework above assumes we can generate perfect samples of the random 
variable $P$. In practice, for complex problems such as SPDEs this is typically impossible.
Instead, we must compute numerical approximations which we will denote by $P_L$, where $L$ 
represents a level of discretisation. For example, and as will be used later,
in a finite difference scheme $L$ could correspond to a mesh with spatial grid spacing 
$h_L = 2^{-(L+1)}$. A higher $L$ means a finer mesh and a more accurate - but also more 
computationally expensive - approximation. 

This introduces a second source of error. The total error of our estimate is now a combination 
of the statistical error from the Monte Carlo sampling and the systematic bias from
numerical discretisation. Consequently, the MSE of the standard MC estimator using 
$N$ samples of the numerical approximation $P_L$, denoted $\hat{P}_{MC, L}$ is:

\begin{equation}\label{eq:mc_mse}
    \text{MSE}(\hat{P}_{MC,L}) = \underbrace{\frac{1}{N}\mathbb{V}[P_L]}_{\substack{\text{Statistical Error} \\ \text{(Variance)}}} + 
    \underbrace{(\mathbb{E}[P_L] - \mathbb{E}[P])^2}_{\substack{\text{Discretisation Error} \\ (\text{Bias}^2)}}
\end{equation}

\eqref{eq:mc_mse} presents the clear dilemma of the standard MC method. To achieve an overall
MSE less than a tolerance  $\varepsilon^2$, both terms must be sufficiently small. 
Reducing the statistical error requires a large number of samples $N$, while 
reducing the discretisation error requires a fine discretisation lvel $L$.
Since the computational cost per sample $C_L$, increases sharply with $L$, the total 
cost, $N \times C_L$, often becomes prohibitively large. 

This is the fundamental challenge that the Multilevel Monte Carlo method is designed to overcome. 
Instead of estimating the expensive quantity $\mathbb{E}[P_L]$ directly, MLMC reformulates it 
using a telescoping sum:
\begin{equation*}
    \mathbb{E}[P_L] = \sum_{\ell=0}^L \mathbb{E}[Y_\ell], \quad \text{where} \quad 
    Y_\ell := P_\ell - P_{\ell-1} \quad \text{and} \quad P_{-1} := 0.
\end{equation*}

Each correction term $\mathbb{E}[Y_\ell]$ is then estimated independently with a 
standard Monte Carlo estimator.

\begin{definition}[Multilevel Monte Carlo Estimator]\label{def:mlmc_estimator}
    Let $Y_\ell = P_\ell - P_{\ell-1}$ be the correction at level $\ell$. The \textbf{Multilevel Monte Carlo (MLMC) estimator}, $\hat{P}_{\mathrm{MLMC}}$, for the expectation $\mathbb{E}[P_L]$ is:
    \[
    \hat{P}_{\mathrm{MLMC}} = \sum_{\ell=0}^L \hat{Y}_\ell, \quad \text{where} \quad \hat{Y}_\ell = \frac{1}{N_\ell} \sum_{n=1}^{N_\ell} Y_\ell^{(n)}.
    \]
\end{definition}


By the linearity of expectation, $\hat{P}_{\mathrm{MLMC}}$ is an unbiased estimator for 
$\mathbb{E}[P_L]$. Since the estimates at each level are independent, its variance is the sum 
of the individual variances. We define the cost and variance of a single sample at 
level $\ell$ as $C_\ell$ and $V_\ell$ respectively:

\begin{align*}
    C_\ell &:= \text{Cost}(Y_\ell) \\
    V_\ell &:= \mathbb{V}[Y_\ell] = \mathbb{V}[P_\ell - P_{\ell-1}] \\
    &= \mathbb{V}[P_\ell] + \mathbb{V}[P_{\ell - 1}]  - 2 \mathrm{Cov}(P_\ell, P_{\ell - 1}).
\end{align*}

The total cost and variance of the MLMC estimator are therefore:
$$
C_{\mathrm{MLMC}} = \sum_{\ell=0}^L N_\ell C_\ell, \qquad 
\mathbb{V}[\hat{P}_{\mathrm{MLMC}}] = \sum_{\ell=0}^L \frac{V_\ell}{N_\ell}.
$$
The success of the MLMC method hinges on the behaviour of the level variances, 
$V_\ell$. The key is to ensure that $V_\ell$ decreases rapidly as the level $\ell$ 
(and therefore cost $C_\ell$) increases. This is achieved by using the same underlying source of 
randomness to generate pairs of samples $(P_\ell, P_{\ell-1})$. This technique, known as
\textit{coupling}, ensures the samples are strongly correlated. Because 
$P_\ell$ and $P_{\ell - 1}$ are approximations of the same underlying quantity, their difference 
is small, and consequently the variance of this difference, $V_\ell$, is much smaller than the 
variance of either term individually. As the level $\ell$ increases, $P_\ell$ converges to 
$P_{\ell - 1}$, and so we expect that $V_\ell \to 0$.

This decay of the level variance allows for a crucial trade-off: we use a small number 
of samples $N_\ell$ for the expensive, high-level correction terms 
(where $V_\ell$ is small) and compensate by using a large number of samples 
for the cheap, low-level terms where the variance is high. The optimal allocation 
of samples across levels can be determined by solving a constrained optimisation problem.
For a fixed variance, $\varepsilon^2$, choosing the optimal $\{N_\ell\}_{\ell=0}^L$ that 
minimises the total cost $C_{\textrm{MLMC}} = \sum N_\ell C_\ell$ is solveable with 
Lagrange mutipliers. It yieds the optimal number of samples \cite{giles2015multilevel}:

\begin{equation}\label{eq:optimal_N_l}
    N_\ell = \left\lceil \frac{1}{\varepsilon_{\text{var}}^2} \sqrt{\frac{V_\ell}{C_\ell}} \sum_{k=0}^L \sqrt{V_k C_k} \right\rceil.
\end{equation}
The ceiling function $\lceil \cdot \rceil$ ensures the number of samples is an integer.
Equation \eqref{eq:optimal_N_l} states that we should take more samples when the 
variance per unit cost of a level is high, and less when it is low.


We also have that the bias $(\mathbb{E}[P_\ell] - \mathbb{E}[P]) \to 0$ as $\ell \to \infty$. 
To ensure that MSE is less than $\varepsilon^2$, by Definition \ref{def:mse_rmse} we can 
impose that 
$(\mathbb{E}[P_L  - P])^2 < \frac{\varepsilon^2}{2}$ and 
$\mathbb{V}[\hat{P}_{\mathrm{MLMC}}] < \frac{\varepsilon^2}{2}]$. 

This leads to the following theorem \cite{giles2015multilevel} which makes 
precise the cost scaling of the MLMC method:

\begin{theorem}[MLMC Complexity Theorem]\label{theorem:mlmc_complexity}
    Let $P$ denote a random variable, and let $P_\ell$ denote the corresponding 
    level $\ell$ numerical approximation. 

    If there exists independent estimators $Y_\ell$ based on $N_\ell$ Monte Carlo
    samples, each with expected cost $C_\ell$ and variance $V_\ell$, and 
    positive constants $\alpha, \beta, \gamma,c_1, c_2, c_3$ such that 
    $\alpha \geq \frac{1}{2} \min(\beta, \gamma)$ and we have
    \begin{enumerate}
        \item \textbf{Weak Error (Bias) Decay: } $|\mathbb{E}[P_\ell - P]| \leq c_1 2^{-\alpha \ell}$,
        \item \textbf{Unbiased Estimators: } $\mathbb{E}[Y_\ell] = 
        \begin{cases}
            E[P_0], & l = 0 \\
            E[P_l - P_{l-1}], & l > 0,
        \end{cases}
        $
        \item \textbf{Variance Decay: } $V_\ell \leq c_2 2^{-\beta \ell}$,
        \item \textbf{Cost Growth: } $C_\ell \leq c_3 2^{\gamma \ell}$,
    \end{enumerate}
    then there exists a positive constant $c_4$ such that for any $\varepsilon < e^{-1}$ there
    are values $L$ and $N_L$ for which the multilevel estimator 
    \begin{equation*}
        Y = \sum_{\ell = 0}^L Y_\ell,
    \end{equation*}
    has an MSE with bound 

    \begin{equation}
        \text{MSE} \equiv \mathbb{E}\left[(Y - \mathbb{E}[P])^2\right] < \varepsilon^2
    \end{equation}

    with a computational complexity $C$ with bound

    \begin{equation}
        \mathbb{E}[C] \leq
        \begin{cases}
            c_4 \varepsilon^{-2}, \qquad &\beta > \gamma,\\
            c_4\varepsilon^{-2}(\log \varepsilon)^2, \qquad &\beta = \gamma,\\
            c_4\varepsilon^{-2-(\gamma - \beta)/\alpha}, ]\qquad &\beta < \gamma
        \end{cases}
    \end{equation}
\end{theorem}

To appreciate the significance of the MLMC Complexity Theorem, we first establish the cost of the 
standard MC method in the above context. To achieve an MSE of 
$\mathcal{O}(\varepsilon^2)$, both the statistical and discretisation errors must be controlled.
Controlling the \textbf{bias} to $O(\varepsilon)$ requires using a fine grid with a 
step size $h_L \propto \varepsilon^{1/\alpha}$. Independently, controlling the 
statistical error to $O(\varepsilon^2)$ requires 
$N \propto \varepsilon^{-2}$ samples. The total cost is the product of the number of 
samples and the cost per sample, where $C_L \propto h_L^{-\gamma}$. Combining these 
requirements gives the overall complexity:

$$
C_{\mathrm{MC}} \propto N \times C_L \propto \varepsilon^{-2} 
\times (\varepsilon^{1/\alpha})^{-\gamma} = \varepsilon^{-2-\gamma/\alpha}.
$$

In contrast, in the MLMC case $\beta > \gamma$, the dominant computational
cost is on coarsest level where the cost per sample $C_\ell$ is $O(1)$ 
\cite{giles2015multilevel}.
Requiring $N = O(\varepsilon^{-2})$ samples provides the dominant cost. This is the optimal 
case.

When $\beta = \gamma$, the cost contribution from each level, 
$N_\ell C_\ell \propto \sqrt{V_\ell C_\ell}$, is approximately constant across 
all levels \cite{giles2015multilevel}. The total cost is therefore proportional to the 
number of levels, $L$, which must increase as $\mathcal{O}(\log \varepsilon)$ to 
meet the bias requirement. This results in the total complexity of 
$\mathcal{O}(\varepsilon^{-2}(\log \varepsilon)^2)$, which remains a vast 
improvement over the standard method.

In the case $\beta < \gamma$, the cost per level grows with $\ell$, meaning 
the total cost is dominated by the work on the finest level, $L$. 
Even in this worst case scenario though, we still arrive at a smaller scaling of cost 
than the MC estimator. 

We conclude this section by highlighting the importance of the relationship
between variance decay rate $\beta$, and the cost growth rate $\gamma$. 
As shown in the Complexity Theorem, the magnitude of the computational savings 
offered by an MLMC implementation depends critically on whether the variance 
decreases faster than the cost increases (i.e. if $\beta \ge \gamma$). 
Determining this relationship for specific SPDE applications and coupling 
strategies is a primary goal of this dissertation.

\section{Introduction}

Stochastic partial differential equations (SPDEs) commonly arise in the mathematical modelling of systems 
whose evolution is significantly influenced by random fluctuations. They have been employed in models of turbulence,
population dynamics, neurophysiology, finance, fluid dynamics and climate modelling \cite{pardoux2021stochastic}
\cite{alonso2020modelling}. Solving SPDEs is often inherently challenging however. Very few cases have analytical
solutions, and numerical solutions have their own obstacles.
Additional dimensionality imposes a higher computational cost, one may have to 
sample over many realisations of the noise, and some SPDEs are highly singular.

The Monte Carlo method is a natural avenue of approach when trying to solve SPDEs. Repeated samples of 
finite-difference or finite-element derived quantities may be obtained, and an estimate of a 
desired quantity of interest therefore computed. They can be exhoribitantly costly however. To achieve an 
accuracy of $\epsilon$ in one's estimate, the standard Monte Carlo
estimator requires $O(\epsilon^{-2})$ samples be computed. To improve on this a number
of variance reduction techniques have been introduced, their aim being to achieve an equivalent accuracy 
at a reduced cost.

The Multilevel Monte Carlo (MLMC) method is one such technique. Its tenet is to sample from a blend of 
cheap, less accurate samples and expensive, more accurate samples, and then combine those to obtain 
one's estimate. Provided this is implemented apprpriately, this can result in a cost reduction when compared
to standard Monte Carlo. 

This dissertation investigates applying the MLMC to solve SPDEs. In particular,
this dissertation has the following goals:
\begin{itemize}
    \item Quantify the cost reductions achievable with MLMC methods, using MC methods as a baseline.
    \item Explore variations in MLMC implementations across different SPDEs.
    \item Present performance improvements achievable with high-performance implementations on GPUs
\end{itemize}


This dissertation will have the following structure. Section ?? will cover the relevant prerequisites. It will 
give an overview of the MLMC method as it pertains to SPDEs, before illustrating with a simple example. A review 
of the relevant literature will then be given. Finally, some relevant definitions will be covered. Section ??
will then begin by introducing the implemented MLMC process. Much of the section is given over to validation 
of quantities, with particular focus given to the stochastic heat equation. Finally we go onto investigating results.


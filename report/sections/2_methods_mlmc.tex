\section{MLMC Algorithm}\label{sec:mlmc_algorithm}

Having outlined the MLMC theory in Section \ref{sec:intro_mlmc}, we now present 
the MLMC algorithm that is implemented in this dissertation. This is 
based on the MLMC implementation provided by Giles \cite{giles2015multilevel}.
The practical implementation of the MLMC method follows a two-phase approach.
The first phase is dedicated to empirically estimating the parameters 
of the MLMC Complexity Theorem \ref{theorem:mlmc_complexity}, while the
second phase uses these paramaters in an adaptive algorithm 
to achieve a final result.

Before running the main adaptive algorithm, A fixed, large number of $N$ 
samples is simulated on a specified range of discretisation levels. From 
these simulations we gather empirical estimates of $|\mathbb{E}[Y_\ell]|, 
V_\ell, C_\ell$. Performing linear regression on the logarithms of these
quantities against the index level we obtain estimates for the weak error 
decay rate, variance decay rate, and cost growth rate, $\alpha$, $\beta$ and $\gamma$ 
respectively. 

Those estimates can then be passed to the MLMC algorithm presented here.

\begin{algorithm}
\caption{Adaptive Multilevel Monte Carlo Algorithm}
\label{alg:mlmc_detailed}
\begin{algorithmic}[1]
\Statex \textbf{Input:} Target RMSE $\varepsilon > 0$, initial number of samples
$N_0$, $\alpha$, $\beta$ and $\gamma$ estimates.
\Statex \textbf{Output:} MLMC estimate $\hat{P}_{\mathrm{MLMC}}$, sample counts
$N_\ell$, variances $V_\ell$, and costs $C_\ell$ for each level.

\State \textbf{Initialize:}
\State Set finest level $L \gets 2$.
\State Set initial required samples $\Delta N_\ell \gets N_0$ for $\ell = 0, 
\dots, L$.
\State Set current sample counts $N_\ell \gets 0$ for $\ell = 0, \dots, L$.
\State Initialize statistical accumulators for sums of $Y_\ell$ and $Y_\ell^2$.

\State
\While{$\sum_{\ell=0}^L \Delta N_\ell > 0$} \Comment{Loop until all levels have
sufficient samples}
    \State \Comment{\textit{1: Generate Samples}}
    \ForAll{$\ell = 0, \dots, L$}
        \If{$\Delta N_\ell > 0$}
            \State Simulate $\Delta N_\ell$ new samples of the correction term 
            $Y_\ell = P_\ell - P_{\ell-1}$.
            \State Update the statistical accumulators (sums of $Y_\ell$ and 
            $Y_\ell^2$).
            \State Update the current sample count: $N_\ell \gets N_\ell + 
            \Delta N_\ell$.
        \EndIf
    \EndFor
    \State
    \State \Comment{\textit{2: Update Estimates and Optimal Allocation}}
    \ForAll{$\ell = 0, \dots, L$}
        \State Compute the current estimate of the variance $V_\ell \gets 
        \mathbb{V}[Y_\ell]$ from the accumulators.
    \EndFor
    \State Calculate the optimal number of samples $N_\ell^{\text{opt}}$ for all 
    levels using the current variance estimates $V_\ell$ and the cost per sample 
    $C_\ell$ in equation \eqref{eq:optimal_N_l} with a target statistical variance of $\varepsilon^2/2$.
    \State Compute the required additional samples: $\Delta N_\ell \gets \max(0, 
    N_\ell^{\text{opt}} - N_\ell)$.
    \State
    \State \Comment{\textit{3: Check for Bias Convergence}}
    \If{$\sum \Delta N_\ell < 0.01 \sum N_\ell$} \Comment{Check only if sampling
    has stabilized}
        \State Compute the mean of the finest correction, $|\hat{Y}_L| = 
        \left|\frac{1}{N_L}\sum_{n=1}^{N_L}Y_L^{(n)}\right|$.
        \If{$|\hat{Y}_L| \ge \varepsilon/\sqrt{2}$} \Comment{If bias is still 
        too large}
            \State $L \gets L+1$. \Comment{Add a new, finer level.}
            \State Initialize accumulators, $N_L$, and $\Delta N_L$ for the 
            new level.
        \EndIf
    \EndIf
\EndWhile

\State
\State \textbf{Compute Final Estimate:} $\hat{P}_{\mathrm{MLMC}} \gets 
\sum_{\ell=0}^L \left( \frac{1}{N_\ell} \sum_{n=1}^{N_\ell} Y_\ell^{(n)} \right)$.
\end{algorithmic}
\end{algorithm}

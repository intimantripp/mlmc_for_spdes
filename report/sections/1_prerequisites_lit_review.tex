\section{Literature Review}

The MLMC method was formally introduced and popularised by 
Giles for SDE path simulation in 2008 \cite{giles2008multilevel}, 
building on earlier foundational work on multilevel integration by 
Heinrich starting in 1998 \cite{heinrich1998monte}. Early works focussed 
predominantly on applications to SDEs, particularly in computational 
finance. Other research extended the method's application to a wider range 
of topics, including various classes of SDEs \cite{abdulle2013stabilized,
rhee2015unbiased}, Lévy processes \cite{giles2017multilevel}, 
Numerical Linear Algebra \cite{acebron2020probabilistic},
and Reliability theory \cite{aslett2017multilevel}.

MLMC for SPDEs is a more recent and active area of research. 
A search on Scopus \cite{scopus} of "Multilevel Monte Carlo Stochastic
Partial Differential Equations" at the time of writing returns 88 documents. Changing 
this to "Multilevel Monte Carlo Parabolic Stochastic Partial Differential Equations" refines 
this down to 18. Of the work examining MLMC for SPDEs, much of the focus has been on 
elliptic SPDEs (for example \cite{abdulle2013multilevel, kornhuber2014multilevel, luo2019multilevel}).

This dissertation is concerned with parabolic SPDEs, for which the literature 
provides a smaller but highly relevant set of foundational papers. This review will focus on 
three key works that inform the central questions of this research: establishing the theoretical 
basis for MLMC's efficiency, demonstrating its practical validation, and exploring 
its application to highly singular equations with different coupling strategies. 


Barth, Lang, and Schwab \cite{barth2013multilevel} provide a foundational 
analysis of the convergence and complexity of the MLMC method for a general class of 
parabolic SPDEs. Using a Galerkin method in space and an Euler-Maruyama scheme in time, they 
prove that the  MLMC estimator significantly reduces the computational work required 
to achieve a given accuracy compared to a standard single-level method. Their 
key result shows that the computational complexity can be reduced from $O(h_L^{-(d+4)})$
for a standard MC method to nearly $O(h_L^{-(d+2)})$ for MLMC, where $d$ is the 
spatial dimension and $h_L$ is the finest mesh width. Their work provides the
theoretical underpinning for the cost savings this dissertation seeks to quantify for 
the stochastic heat and Dean-Kawasaki equations. However, their analysis does not explore
the practical performance improvements of different noise coupling strategies.

In \cite{giles2012stochastic}, Giles and Reisinger provide a practical demonstration of 
MLMC's performance improvements for a class of parabolic SPDEs arising in financial modelling. 
The authors develop and analyse a Milstein finite difference scheme, proving it converges with 
first-order accuracy in time and second-order in space. They demonstrate a concrete reduction
in computational complexity from $O(\varepsilon^{-7/2})$ for a standard MC approach to the optimal 
$O(\varepsilon^{-2})$ for their MLMC implementation, validating this gain through 
numerical experiments. This work serves as a methodological benchmark, demonstrating how to 
empirically confirm the theoretical performance gains of the MLMC method.

A very recent and highly relevant contribution is the 2024 paper by 
Cornalba and Fischer, which 
develops and analyses an MLMC method specifically for the Dean-Kawasaki equation, 
one of the two case studies in this dissertation. Their work tackles a highly singular SPDE 
for which standard convergence proofs fail. By formulating their analysis in terms 
of the convergence of probability distributions, they prove that MLMC provides a 
significant computational improvement over standard MC, provided average 
particle density is sufficiently large. Crucially, they propose and 
analyse two distinct noise coupling strategies:
a "Fourier coupling" and a "Right-Most Nearest Neighbours (NN) coupling". We build directly
on this work in this dissertation by also investigating the Dean-Kawasaki equation and proposing 
an alternative noise coupling strategy.


Beyond theoretical convergence and numerical validation, an important 
direction of research examines how to parallelise MLMC effectively on 
high performance computing (HPC) systems. Like standard Monte Carlo, 
MLMC methods naturally lend themselves to parallelisation due to the 
independence of samples. Parallelism can typically be exploited along 
two dimensions: across independent samples on each level and across levels. 
In the context of SPDEs a third dimension can be added, as deterministic 
PDE solvers for each sample can also be parallelised. We examine several 
pieces of relevant work that have considered different aspects of 
HPC implementation for MLMC methods for SPDEs.

Drzisga et al.  \cite{drzisga2017scheduling} analyse how to combine 
these layers of parallelism efficiently on petascale systems, examining 
an elliptic SPDE. They 
introduce the concept of \textit{scalability windows}, showing 
that solver performance depends strongly on the number of 
processors allocated per sample. They classify 
scheduling strategies, including sample-synchronous (all groups 
calculate samples in lockstep), level-synchronous (groups complete 
levels in lockstep), homogenous (all processors work on the 
same level at once), and heterogenous (different processor 
groups work on different levels concurrently). They demonstrate that 
careful scheduling is crucial to achieving high utilisation, and report 
MLMC simulations with over $7 \times 10^{10}$ unknowns 
on 131,072 cores in under 1.5 hours.

In Baumgarten, Krumscheid, and Wieners \cite{baumgarten2024fully}, the 
authors propose a budgeted MLMC (BMLMC) method tailored for HPC environments. 
Instead of fixing an error tolerance, BMLMC minimises the mean square error 
subject to a fixed computational budget, formulated as a knapsack problem 
and solved by dynamic programming. To exploit parallelism effectively they 
introduced a multi-sample finite element method (MS-FEM), which unifies 
solver- and sample-parallelism in a single distributed data structure. 
They establish new error--cost bounds that account for parallelisation bias, 
and demonstrate scalability on thousands of cores for large-scale acoustic 
wave propagation with random media, a parabolic-type PDE closely related 
to the equations considered in this dissertation.

Finally, Iliev et al.\ \cite{iliev2021parallel} 
investigate dynamic scheduling strategies for MLMC in porous media flow, 
emphasising how sample-to-sample variability can harm utilisation. They 
compared job-queue and interruption-based schedulers, showing that dynamic 
strategies maintain efficiency above 0.89 even when scaling to several 
hundred cores. Their results underline the importance of dynamic load 
balancing for MLMC applied to SPDEs on parallel hardware.



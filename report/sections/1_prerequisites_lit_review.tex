\section{Literature Review}

The MLMC method was formally introduced and popularised by 
Giles for SDE path simulation in 2008 \cite{giles2008multilevel}, 
building on earlier foundational work on multilevel integration by 
Heinrich starting in 1998 \cite{heinrich1998monte}. Early works focussed 
predominantly on application to SDEs, particularly in computational 
finance. Other research extended the method's application to a wider range 
of topics, including various classes of SDEs \cite{abdulle2013stabilized,
rhee2015unbiased}, Lévy processes \cite{giles2017multilevel}, 
Numerical Linear Algebra \cite{acebron2020probabilistic},
and Reliability theory \cite{aslett2017multilevel}.

MLMC for SPDEs is a more recent and active area of research. 
A search on Scopus \cite{scopus} of "Multilevel Monte Carlo Stochastic
Partial Differential Equations" at the time of writing returns 88 documents. Changing 
this to "Multilevel Monte Carlo Parabolic Stochastic Partial Differential Equations" refines 
this down to 18. Of the work examining MLMC for SPDEs, much of the focus has been on 
elliptic SPDEs (for example \cite{abdulle2013multilevel, kornhuber2014multilevel, luo2019multilevel}).

This dissertation is concerned with parabolic SPDEs, for which the literature 
provides a smaller but highly relevant set of foundational papers. This review will focus on 
three key works that inform the central questions of this research: establishing the theoretical 
basis for MLMC's efficiency, demonstrating its practical validation, and exploring 
its application to highly singular equations with different coupling strategies. 


Barth, Lang, and Schwab \cite{barth2013multilevel} analyse provide a foundational 
analysis of the convergence and complexity of the MLMC method for a general class of 
parabolic SPDEs. Using a Galerkin method in space and a Euler-Maruyama scheme in time, they 
prove that the  MLMC estimator significantly reduces the computational work required 
to achieve a given accuracy compared to a standard single-level method. Their 
key result shows that the computational complexity can be reduced from $O(h_L^{-(d+4)})$
for a standard MC method to nearly $O(h_L^{-(d+2)})$ for MLMC, where $d$ is the 
spatial dimension and $h_L$ is the finest mesh width. Their work provides 
theoretical underpinning for the cost savings this dissertation seeks to validate for 
the stochastic heat and Dean-Kawasaki equations. However, their analysis does not explore
the practical performance improvements of different noise coupling strategies.

In \cite{giles2012stochastic}, Giles and Reisinger provide a practical demonstration of 
MLMC's performance improvements for a class of parabolic SPDEs arising in financial modelling. 
The authors develop and analyse a Milstein finite difference scheme, proving it converges with 
first-order accuracy in time and second-order in space. They demonstrate a concrete reduction
in computational complexity from $O(\varepsilon^{-7/2})$ for a standard MC approach to the optimal 
$O(\varepsilon^{-2})$ for their MLMC implementation, validating this gain through 
numerical experiments. This work serves as a methodological benchmark, demonstrating how to 
empirically confirm the theoretical performance gains of the MLMC method.

A very recent and highly relevant contribution is the 2024 paper by 
Cornalba and Fischer, which 
develops and analyses an MLMC method specifically for the Dean-Kawasaki equation, 
one of the two case studies in this dissertation. Their work tackles a highly singular SPDE 
for which standard convergence proofs fail. By formulating their analysis in terms 
of the convergence of probability distributions, they prove that MLMC provides a 
significant computational improvement over standard MC, provided average 
particle density is sufficiently large. Crucially, they propose and 
analyse two distinct noise coupling strategies:
a "Fourier coupling" and a "Right-Most Nearest Neighbours (NN) coupling". We build directly
on this work in this dissertation by also investigating the Dean-Kawasaki equation and proposing 
an alternative noise coupling strategy.

NEED TO ADD REGARDING HPC MLMC IMPLEMENTATIONS AND SPDES (believe none done)





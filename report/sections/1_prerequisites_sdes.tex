\section{Stochastic Partial Differential Equations}

In their most general sense, an SPDE is a partial differential equation
where at least one of the following is random: coefficients, 
initial boundary conditions, the domain, and the forcing term 
\cite{lototsky2017stochastic}. In our case, 
the SPDEs we examine have only a random forcing term. 
In this section, we present the two SPDEs that will be
the use cases for our MLMC implementations and investigations.
In section SECTION HERE we will derive and define the finite difference schemes
based on these implementations.


\subsection{The Stochastic Heat Equation}

The stochastic heat equation (SHE) is a canonical example of a parabolic SPDE.
Formally, it is the standard heat equation perturbed by a stochastic
forcing term introducing spatially and temporally uncorrelated 
fluctuations. For instance, heat diffusing through a metal bar 
that experiences heat emitting chemical reactions is a typical scenario the SHE
describes.

The equation is fully specified by the SPDE itself, initial conditions and 
a set of bondary conditions \cite{lototsky2017stochastic,pardoux2021stochastic}.

\begin{align}\label{eq:she_spde}
\frac{\partial u(t,\mathbf{x})}{\partial t} &= \Delta u(t,\mathbf{x}) + \xi(t,\mathbf{x}), \quad \text{for } (t, \mathbf{x}) \in (0, T] \times \Omega \tag{SHE} \\
u(0, \mathbf{x}) &= u_0(\mathbf{x}), \quad \text{for } \mathbf{x} \in \Omega \nonumber \\
\mathcal{B}u(t, \mathbf{x}) &= g(t, \mathbf{x}), \quad \text{for } (t, \mathbf{x}) \in (0, T] \times \partial D \nonumber
\end{align}

Where $u(t,\mathbf{x})$ is a real valued function at time $t \in [0, T]$ and 
spatial position $\mathbf{x}$ within domain $\Omega \subset \mathbb{R}^d$. 
$u_0(\mathbf{x})$ is the initial state of the field,
the operator $\mathcal{B}$ represents the boundary condition on the boundary of domain 
$\partial \mathcal{B}$, and $\Delta$ is the standard Laplacian operator.

The defining component is the stochastic forcing term $\xi(t,\mathbf{x})$ term which denotes 
space-time white noise. We formally define this as follows (\cite{walsh2006introduction}, Chapter 1).

\begin{definition}[Space-Time White Noise]
\label{def:whitenoise}
A space-time white noise $\xi(t,\mathbf{x})$ on $[0,T] \times D$ is a centered
Gaussian process defined by a collection of random variables 
$\{W(\phi)\}$ indexed by test functions $\phi \in L^2([0,T] \times D)$, 
with a covariance structure given by:
\[
\mathbb{E}[W(\phi)W(\psi)] = \langle \phi, \psi \rangle_{L^2}
\]
\end{definition}

This formal defintion has several important consequences. Firstly, it gives rise 
to the more intuitive heuristic covariance expression 
$\mathbb{E}\left[\xi(t,\mathbf{x})\xi(s,\mathbf{y}\right] = 
\delta(t-s)\delta(\mathbf{x}-\mathbf{y})$, implying the noise is perfectly uncorrelated 
at every point. Secondly, the abstract process can be understood as the distributional
derivative of a more tangible (though still highly irregular) object
known as a Brownian sheet, i.e. a multidimensional Brownian motion.

Most critically for our purposes, Definition \ref{def:whitenoise} directly informs 
how we discretise the noise term in a numerical scheme. The integral of 
noise over a discrete space-time grid cell, $C_j^n = [t_n, t_{n+1}] \times
[x_j - \frac{\Delta x}{2}, x_j + \frac{\Delta x}{2}]$, is found by choosing the test
function $\phi$ to be the indicator function of that cell, $\phi_j^n(t, \mathbf{x}) 
= \mathbf{1}_{C_j^n}(t, \mathbf{x})$. 

The random variable representing the integrated noise over this cell is therefore 
$W(\phi_j^n)$. From Definition \ref{def:whitenoise}, we know this is a centred Gaussian 
random variable whose variance is given by:

\begin{subequations}\label{eq:white_noise_integral_derivation}
    \begin{align}
        \mathrm{Var}(W(\phi_j^n)) &= \mathbb{E}[W(\phi_j^n)^2] = 
        \langle \phi_j^n, \phi_j^n \rangle_{L^2} \\
        &= \int_0^T \int_D (\mathbf{1}_{C_j^n}(t, \mathbf{x}))^2 
        \mathrm{d}\mathbf{x} \mathrm{d}t \\
        &= \int_{t_n}^{t_{n+1}} \int_{x_j-\frac{\Delta x}{2}}^{x_j+\frac{\Delta x}{2}} 1
        \mathrm{d}\mathbf{x}\mathrm{d}t \\
        &= \text{Area}(C_j^n) = \Delta t \Delta x
    \end{align}
\end{subequations}

Since any centred Gaussian random variable with variance $\delta^2$ can be written as $\sigma Z$ where 
$Z \sim \mathcal{N}(0,1)$. Equations \eqref{eq:white_noise_integral_derivation}
lead to a result that will be used in our finite difference 
implementations:

\begin{equation}\label{eq:white_noise_result}
    \int_{t_n}^{t_{n+1}} \int_{x_j - \frac{\Delta x}{2}}^{x_j + \frac{\Delta x}{2}} \xi(t, \mathbf{x})
    \mathrm{d}x \mathrm{d}t = \sqrt{\Delta t \Delta x} Z_j^n
\end{equation}

where $Z_j^n$ are idependent and identically distirbuted standard normal random variables.

\subsection{The Dean-Kawasaki Equation}
-
The Dean-Kawasaki (DK) equation is used to describe the evolution of the density 
$\rho(\mathbf{x}, t)$ of a system of $N >> 1$ weakly interacting particles, 
having emerged from the field of fluctuating hydrodynamics.
For the non-interacting case investigated in this dissertation, the equation is given by
\cite{cornalba2025multilevel}:

\begin{align}\label{eq:dk_spde_2}
\frac{\partial \rho(t, \mathbf{x})}{\partial t} &=
\frac{1}{2}\Delta\rho(t, \mathbf{x}) + N^{-1/2}\nabla\cdot(\sqrt{\rho(t, \mathbf{x})}\xi(t, \mathbf{x})), 
&& \text{for } (t, \mathbf{x}) \in (0, T] \times \Omega \tag{DK} \\
\rho(0, \mathbf{x}) &= \rho_0(\mathbf{x}), && \text{for } \mathbf{x} \in \Omega \nonumber \\
\rho(t, \mathbf{x}) &\text{ satisfies periodic b.c. on } \partial \Omega, && \text{for } t \in (0, T] \nonumber
\end{align}

Here, $\rho(t, \mathbf{x})$ is the particle density at time $t$ in a 
domain $\Omega = \mathbb{T}^d$, the $d$-dimensional torus. 
The term $\frac{1}{2} \Delta \rho$ describes
typical particle diffusion. The stochastic forcing term $N^{-1/2}
\nabla \cdot (\sqrt{\rho}\xi)$ captures the particle density flux,
where $\xi$ is space-time white noise. Intuitively, this noise scaled
by $\sqrt{\rho(t, \mathbf{x})}$.


The equation is highly singular, and "strong" or "pathwise" solutions 
i.e. solutions that exist as a function of every given realisation 
of the noise, do not exist. Instead, the only existing solutions are martingale 
solutions where $\rho$ corresponds to empirical measures of the underlying 
particle system:

\begin{equation*}
    \rho(\mathbf{x},t) \equiv \mu_t^N(\mathbf{x}) := N^{-1} \sum_{i=1}^N\delta(x - X_i(t)).
\end{equation*}

Cornalba and Fischer \cite{cornalba2025multilevel} do demonstrate however that
statistical properties of fluctuations 
around the mean-field limit $\bar{rho}$ can be simulated. 
Generally, quantities of interest, $Q$, of the form:

\begin{equation} 
    Q = \psi\big(N^{1/2} \int (\mu_N^T - \bar{\rho}^T)(\mathbf{x})
    \phi(\mathbf{x})\mathrm{d}\mathbf{x}\big)
\end{equation}


can be simulated for sufficiently regular test functions $\psi, \phi$, provided $Nh^d >> 1$
(i.e. as long as on average there is more than one particle per grid cell). 
For example, choosing $\psi(z) = z^2$ allows for the computation of the 
variance of the fluctuations.





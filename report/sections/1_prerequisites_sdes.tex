\section{Stochastic Partial Differential Equations}

In their most general sense, an SPDE is a partial differential equation
where at least one of the following is random: coefficients, 
initial boundary conditions, the domain, and the forcing term 
\cite{lototsky2017stochastic}. In our case, 
the SPDEs we examine have only a random forcing term. In this section, 
we will present the two SPDEs we consider, define their relevant 
terms and discuss their specifics.

\subsection{The Stochastic Heat Equation}

The stochastic heat equation (SHE) is a canonical example of a parabolic SPDE.
Formally, it is the standard heat equation perturbed by a stochastic
forcing term introducing spatially and temporally uncorrelated 
fluctuations. For instance, heat diffusing through a metal bar 
that experiences heat emitting chemical reactions is a typical scenario the SHE
describes.

\begin{align}\label{eq:she_spde}
\frac{\partial u(t,\mathbf{x})}{\partial t} &= \Delta u(t,\mathbf{x}) + \xi(t,\mathbf{x})
\tag{SHE}\\
u(0, \mathbf{x}) &= u_0(\mathbf{x})
\end{align}

$u(t,\mathbf{x})$ is a real valued function dependent on time $t \in [0, T]$ and 
spatial position $\mathbf{x}$ within domain $D \subset \mathbb{R}^d$. $\Delta$ is the Laplacian operator.

The defining component is the stochastic forcing term $\xi(t,\mathbf{x})$ term which denotes 
space-time white noise. $\xi$ is not a classical function but a generalised 
stochastic process, or random distribution. It is mmost accurately understand as the 
distributional derivative of a Brownian sheet, $W(t,\mathbf{x})$, which is a centred Gaussian 
process indexed by $(t,\mathbf{x})$ with independent increments over disjoint rectangles in 
space-time. $\xi$ is therefore formally defined by its inner product against a test function $\phi$,
as $\langle\xi, \phi\rangle$. This random variable is Gaussian and defined by the following two properties:

\begin{align}\label{eq:white_noise_defs}
    \text{Zero Mean: } & \mathbb{E}\left[\langle \xi, \phi \rangle \right] = 0 \\
    \text{Covariance Structure: } & \mathbb{E}\left[\langle\xi,\phi\rangle \langle\xi,\psi\rangle\right]
    = \langle\phi, \psi\rangle_{L^2} = \int_0^T \int_D \phi(t, \mathbf{x}) \psi(t, \mathbf{x}) \,\mathrm{d}\mathbf{x} \,\mathrm{d}t
\end{align}

In section SECTION HERE we will derive the finite difference schemes used here for these 
quantities using equations \eqref{eq:white_noise_defs}.


\subsection{The Dean-Kawasaki Equation}

The Dean-Kawasaki (DK) equation is used to describe the evolution of the density 
$\rho(\mathbf{x}, t)$ of a system of $N >> 1$ weakly interacting particles.
For the non-interacting case investigated in this dissertation, the equation is given by:

\begin{align}\label{eq:dk_spde}
\frac{\partial_\rho(t, \mathbf{x})}{\partial t} &= 
\frac{1}{2}\Delta\rho(t, \mathbf{x}) + N^{-1/2}\nabla\cdot(\sqrt{\rho(t, \mathbf{x})}\xi(t, \mathbf{x})) \tag{DK} \
\rho(0, \mathbf{x}) &= \rho_0(\mathbf{x}) \nonumber
\end{align}

Here, $\rho(t, \mathbf{x})$ is the particle density at time $t$ and position $\mathbf{x}$.
The term $\frac{1}{2} \Delta \rho$ describes standard particle diffusion. 
The stochastic forcing term $N^{-1/2} \nabla \cdot (\sqrt{\rho}\xi)$ models the density 
fluctuations, where $\xi$ is space-time white noise. 

The $DK$ equation presents significant mathematical and numerical challenges. The noise term 
is multiplicative, its magnitude scaled by the local density through $\sqrt{\rho}$.
The equation is highly singular. As shown in \cite{konarovskyi2019dean}, the only 
martingale solutions to DK are empirical measures of the underlying particle system. Thus
we require

\begin{equation*}
    \rho(\mathbf{x},t) \equiv \mu_t^N(\mathbf{x}) := N^{-1} \sum_{i=1}^N\delta(x - X_i(t)).
\end{equation*}

Cornalba and Fischer \cite{cornalba2025multilevel} however do demonstrate that
statistical properties of fluctuations 
around the mean-field limit $\bar{rho}$ can be simulated. 
Generally, quantities of interest of the form:

\begin{equation}
    Q = \psi\big(N^{1/2} \int (\mu_N^T - \bar{\rho}^T)(\mathbf{x})
    \phi(\mathbf{x})\mathrm{d}\mathbf{x}\big)
\end{equation}


This can correspond to for example the variances 
$\mathbb{E}[|N^{1/2}\int(\\mu_T^N - \bar{\rho}^T)(\mathbf{x})\phi(\mathbf{x})\mathrm{d}x|^2]$.




